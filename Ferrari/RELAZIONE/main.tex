\documentclass{article}
\usepackage{xcolor}
\usepackage{listings}
%% FILENAME: defs.tex
%% AUTHOR:   Cameron Swords

\usepackage{amsmath, listings, amsthm, amssymb, proof, xspace}

%%------------------------------------------------------------------------
%% DEFINITION HELPERS
%%------------------------------------------------------------------------

\newcommand{\alt}{~~|~~}
\newcommand{\comp}[1]{\llbracket #1 \rrbracket}

\newcommand{\inlinexp}[1]{
{\footnotesize
 \[\begin{array}{l}
 #1
 \end{array}\]}}

\newcommand{\inlinexpa}[2]{
{\footnotesize
 \[\begin{array}{#1}
 #2
 \end{array}\]}}

\newcommand{\infr} [3] [] {\infer[\textsc{#1}]{#3}{#2}}
\newcommand{\iand}        {\qquad}

\newcommand{\Ctxt}       {\mathcal{E}}
\newcommand{\InCtxt} [1] {\Ctxt[#1]}

%%------------------------------------------------------------------------
%% REDUCTION RELATION MACROS
%%------------------------------------------------------------------------

\newcommand{\subst} [3]    {#3 [#2 / #1]}
\newcommand{\dstep} [2]    {#1 ~\Downarrow~ #2}

\newcommand{\ssosredex}        {\rightarrow}
\newcommand{\ctxtreduce}       {\mapsto}
\newcommand{\sstep}     [3] [] {#2 &\ssosredex&  #3 &\textsc{#1}}
\newcommand{\ctxtstep}  [3] [] {#2 &\ctxtreduce& #3 &\textsc{#1}}

%%------------------------------------------------------------------------
%% TYPE DEFINITION MACROS
%%------------------------------------------------------------------------

\newcommand{\funct} [2] {#1\nobreak\rightarrow\nobreak#2}
\newcommand{\boolt}     {\mathtt{bool}}

\newcommand{\typeEnv}         {\Gamma}
\newcommand{\entails}         {\vdash}
\newcommand{\judgment}   [3] {#1 \entails #2 : #3}
\newcommand{\envent}      [2] {\judgment{\typeEnv}{#1}{#2}}
\newcommand{\extenvent}   [4] {\judgment{\typeEnv, #1 : #2}{#3}{#4}}
\newcommand{\envlookup}   [3] {\infr{#1(#2) = #3}{\judgment{#1}{#2}{#3}}}

%%------------------------------------------------------------------------
%% EXPRESSION MACROS
%%------------------------------------------------------------------------

%% lambda
\newcommand{\lamdefe}  [2] {\lambda #1.~#2}
\newcommand{\lamdefea} [2] {\begin{array}{l}\lambda#1.\\\hspace*{.5em}#2\\\end{array}}

%% let
\newcommand{\letdefe}    [3] {\letbind{#1}{#2}~\letin{#3}}
\newcommand{\letdefarre} [3] {\begin{array}{l}\letbind{#1}{#2}\\\letin{#3})\\\end{array}}

\newcommand{\letbind}  [2] {\mathsf{let}~\lbind{#1}{#2}}
\newcommand{\letbindp} [2] {\mathsf{let}~(\lbind{#1}{#2})}
\newcommand{\lbind}    [2] {#1=#2}
\newcommand{\letin}    [1] {\mathsf{in}~#1}

%% if
\newcommand{\ife}      [3] {\ifline{#1}~\thenline{#2}~\elseline{#3}}
\newcommand{\ifea}     [3] {\begin{array}{l}\ifline{#1}\\\thenline{#2}\\\elseline{#3}\end{array}}

\newcommand{\ifop}         {\mathsf{if}}
\newcommand{\ifline}   [1] {\ifop~ #1}
\newcommand{\thenline} [1] {\mathsf{then}~#1}
\newcommand{\elseline} [1] {\mathsf{else}~#1}

%% opers
\newcommand{\binopdef}     {\mathit{binop}}
\newcommand{\unopdef}      {\mathit{unop}}
\newcommand{\binope}   [2] {\binopdef~#1~#2}
\newcommand{\unope}    [1] {\unopdef~#1}

\newcommand{\andop}        {\mathsf{and}}
\newcommand{\orop}         {\mathsf{or}}
\newcommand{\notop}        {\mathsf{not}}
\newcommand{\ande}     [2] {\mathsf{and}~#1~#2}
\newcommand{\ore}      [2] {\mathsf{or}~#1~#2}
\newcommand{\note}     [1] {\mathsf{not}~#1}

%% values
\newcommand{\falsev}     {\mathsf{false}}
\newcommand{\truev}      {\mathsf{true}}

\lstset{basicstyle=\ttfamily,
  showstringspaces=false,
  commentstyle=\color{red},
  keywordstyle=\color{blue}
}
\usepackage[utf8]{inputenc}
\title{%
  System And Languages for Informatics \\
  \large Cybersecurity LM-66}

\author{Samuele Padula}
\date{Gennaio 2021}

\usepackage{natbib}
\usepackage{graphicx}

\begin{document}

\maketitle

\section{Introduzione}
È stato realizzato un interprete in OCaml per un ridotto linguaggio di programmazione funzionale. L’intero interprete con solamente typechecking dinamico e comprensivo dei test effettuati, è contenuto in un unico file sorgente "interprete\_dinamico.ml". Nel file "interprete\_statico.ml" è stato invece sviluppato l'interprete con typechecking statico rimuovendo tutti i controlli di tipo dinamici compiuti in precedenza dalla 'eval'.  L'interprete visto a lezione è stato esteso con la possibilità di creare e manipolare gli insiemi. Gli insiemi possono contenere solamente Interi, Stringhe e Booleani.
Per eseguire la batteria di test si può usare il comando ocaml (il tutto è stato simulato e sviluppato in ambiente Linux Ubuntu 18.04 ocaml 4.05):

\begin{lstlisting}[language=bash]
$ocaml interprete_dinamico.ml
ocaml interprete_statico.ml
\end{lstlisting}

\section{Regole operazionali (Set)}
\subsection{Introduzione del tipo di dato Set}
~\\~\\
\infr
  {\\env \vartriangleright{} e \implies t:String\ \ \ \ t \in \{"int", "bool", "string"\}}     
  {\\env \vartriangleright{} \bold{EmptySet(e)} \implies Set(t,\emptyset) }
~\\~\\~\\~\\
\infr
  {\\env \vartriangleright{} e_1 \implies t:String\ \ \ \ t \in \{"int", "bool", "string"\},  env \vartriangleright{} e_2 \implies v:t}     
  {\\env \vartriangleright{} \bold{Singleton(e_1,e_2)} \implies Set(t,{v}) }  
~\\~\\~\\~\\
\infr
  {\\env \vartriangleright{} e_1 \implies t:String\ \ \ \ t \in \{"int", "bool", "string"\},  env \vartriangleright{} e_2 \implies (v_1...v_n):t}     
  {\\env \vartriangleright{} \bold{Of(e_1,e_2)} \implies Set(t,\{v_1...v_n\}) }  

\subsection{Operazioni sul tipo di dato Set}
\subsubsection{Operazioni di base}
\infr
  {\\env \vartriangleright{} e_1 \implies Set(t,s1), env \vartriangleright{} e_2 \implies Set(t,s2)\ \ \ \ t \in \{"int", "bool", "string"\}}     
  {\\env \vartriangleright{} \bold{Union(e_1,e_2)} \implies Set(t,\{s_1 \cup s_2\}) }  
~\\~\\
\infr
  {\\env \vartriangleright{} e_1 \implies Set(t,s1), env \vartriangleright{} e_2 \implies Set(t,s2)\ \ \ \ t \in \{"int", "bool", "string"\}}     
  {\\env \vartriangleright{} \bold{Intersection(e_1,e_2)} \implies Set(t,\{s_1 \cap s_2\}) }  
~\\~\\
\infr
  {\\env \vartriangleright{} e_1 \implies Set(t,s1), env \vartriangleright{} e_2 \implies Set(t,s2)\ \ \ \ t \in \{"int", "bool", "string"\}}     
  {\\env \vartriangleright{} \bold{Difference(e_1,e_2)} \implies Set(t,\{s_1 \setminus s_2\}) }  
\subsubsection{Operazioni aggiuntive}
\infr
  {\\env \vartriangleright{} e_1 \implies Set(t,s), env \vartriangleright{} e_2 \implies v:t\ \ \ \ t \in \{"int", "bool", "string"\}}     
  {\\env \vartriangleright{} \bold{Insert(e_1,e_2)} \implies Set(t, s \cup \{v\}) }  
~\\~\\
\infr
  {\\env \vartriangleright{} e_1 \implies Set(t,s), env \vartriangleright{} e_2 \implies v:t\ \ \ \ t \in \{"int", "bool", "string"\}}     
  {\\env \vartriangleright{} \bold{Remove(e_1,e_2)} \implies Set(t, s \setminus \{v\}) }  
~\\~\\~\\~\\
\infr
  {\\env \vartriangleright{} e_1 \implies Set(t,s), env \vartriangleright{} e_2 \implies v:t\ \ \ \ t \in \{"int", "bool", "string"\}}     
  {\\env \vartriangleright{} \bold{Contains(e_1,e_2)} \implies Bool(v \in s) }  
~\\~\\~\\~\\
\infr
  {\\env \vartriangleright{} e_1 \implies Set(t,s), env \vartriangleright{} e_2 \implies v:t\ \ \ \ t \in \{"int", "bool", "string"\}}     
  {\\env \vartriangleright{} \bold{Subset(e_1,e_2)} \implies Bool(\forall{x} \in s) }  
~\\~\\~\\~\\
\infr
  {\\env \vartriangleright{} e \implies Set(t,s)\ \ \ \ t \in \{"int", "bool", "string"\}}     
  {\\env \vartriangleright{} \bold{IsEmpty(e)} \implies Bool(s = \emptyset) }  
~\\~\\~\\~\\
\infr
  {\\env \vartriangleright{} e \implies Set(t,s)\ \ \ \ t \in \{"int", "bool", "string"\}}     
  {\\env \vartriangleright{} \bold{MinOf(e)} \implies min(s):t  }
~\\~\\~\\~\\
\infr
  {\\env \vartriangleright{} e \implies Set(t,s)\ \ \ \ t \in \{"int", "bool", "string"\}}     
  {\\env \vartriangleright{} \bold{MaxOf(e)} \implies max(s):t }
~\\~\\~\\~\\
\infr
  {\\env \vartriangleright{} e_1 \implies Set(t,s), env \vartriangleright{} e_2 \implies f: t \xrightarrow{} Boolean\ \ \ \ t \in \{"int", "bool", "string"\}}     
  {\\env \vartriangleright{} \bold{ForAll(e_1,e_2)} \implies Bool(\forall{x} \in s, Apply(e_2,x) \implies Bool(true))}  
~\\~\\~\\~\\
\infr
  {\\env \vartriangleright{} e_1 \implies Set(t,s), env \vartriangleright{} e_2 \implies f: t \xrightarrow{} Boolean\ \ \ \ t \in \{"int", "bool", "string"\}}     
  {\\env \vartriangleright{} \bold{Exists(e_1,e_2)} \implies Bool(\exists{x} \in s, Apply(e_2,x) \implies Bool(true))}  
~\\~\\~\\~\\
\infr
  {\\env \vartriangleright{} e_1 \implies Set(t,s), env \vartriangleright{} e_2 \implies f: t \xrightarrow{} Boolean\ \ \ \ t \in \{"int", "bool", "string"\}}     
  {\\env \vartriangleright{} \bold{Filter(e_1,e_2)} \implies Set(t,\{x \in s, Apply(e_2,x) \implies Bool(true)\}}  
~\\~\\~\\~\\
\infr
  {\\env \vartriangleright{} e_1 \implies Set(t_1,s), env \vartriangleright{} e_2 \implies f: t_1 \xrightarrow{} t_2\ \ \ \  t_1,t_2 \in \{"int", "bool", "string"\}}     
  {\\env \vartriangleright{} \bold{Map(e_1,e_2)} \implies Set(t_2,\{Apply(e_2,x \in s)\}}  
~\\~\\~\\~\\


\end{document}
