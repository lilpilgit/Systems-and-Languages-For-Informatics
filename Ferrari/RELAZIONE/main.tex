\documentclass{article}
\usepackage{xcolor}
\usepackage{listings}
\input{defs.tex}
\lstset{basicstyle=\ttfamily,
  showstringspaces=false,
  commentstyle=\color{red},
  keywordstyle=\color{blue}
}
\usepackage[utf8]{inputenc}
\title{%
  System And Languages for Informatics \\
  \large Cybersecurity LM-66}

\author{Samuele Padula}
\date{Gennaio 2021}

\usepackage{natbib}
\usepackage{graphicx}

\begin{document}

\maketitle

\section{Introduzione}
È stato realizzato un interprete in OCaml per un ridotto linguaggio di programmazione funzionale. L’intero interprete con solamente typechecking dinamico e comprensivo dei test effettuati, è contenuto in un unico file sorgente "interprete\_dinamico.ml". Nel file "interprete\_statico.ml" è stato invece sviluppato l'interprete con typechecking statico rimuovendo tutti i controlli di tipo dinamici compiuti in precedenza dalla 'eval'.  L'interprete visto a lezione è stato esteso con la possibilità di creare e manipolare gli insiemi. Gli insiemi possono contenere solamente Interi, Stringhe e Booleani.
Per eseguire la batteria di test si può usare il comando ocaml (il tutto è stato simulato e sviluppato in ambiente Linux Ubuntu 18.04 ocaml 4.05):

\begin{lstlisting}[language=bash]
$ocaml interprete_dinamico.ml
ocaml interprete_statico.ml
\end{lstlisting}

\section{Regole operazionali (Set)}
\subsection{Introduzione del tipo di dato Set}
~\\~\\
\infr
  {\\env \vartriangleright{} e \implies t:String\ \ \ \ t \in \{"int", "bool", "string"\}}     
  {\\env \vartriangleright{} \bold{EmptySet(e)} \implies Set(t,\emptyset) }
~\\~\\~\\~\\
\infr
  {\\env \vartriangleright{} e_1 \implies t:String\ \ \ \ t \in \{"int", "bool", "string"\},  env \vartriangleright{} e_2 \implies v:t}     
  {\\env \vartriangleright{} \bold{Singleton(e_1,e_2)} \implies Set(t,{v}) }  
~\\~\\~\\~\\
\infr
  {\\env \vartriangleright{} e_1 \implies t:String\ \ \ \ t \in \{"int", "bool", "string"\},  env \vartriangleright{} e_2 \implies (v_1...v_n):t}     
  {\\env \vartriangleright{} \bold{Of(e_1,e_2)} \implies Set(t,\{v_1...v_n\}) }  

\subsection{Operazioni sul tipo di dato Set}
\subsubsection{Operazioni di base}
\infr
  {\\env \vartriangleright{} e_1 \implies Set(t,s1), env \vartriangleright{} e_2 \implies Set(t,s2)\ \ \ \ t \in \{"int", "bool", "string"\}}     
  {\\env \vartriangleright{} \bold{Union(e_1,e_2)} \implies Set(t,\{s_1 \cup s_2\}) }  
~\\~\\
\infr
  {\\env \vartriangleright{} e_1 \implies Set(t,s1), env \vartriangleright{} e_2 \implies Set(t,s2)\ \ \ \ t \in \{"int", "bool", "string"\}}     
  {\\env \vartriangleright{} \bold{Intersection(e_1,e_2)} \implies Set(t,\{s_1 \cap s_2\}) }  
~\\~\\
\infr
  {\\env \vartriangleright{} e_1 \implies Set(t,s1), env \vartriangleright{} e_2 \implies Set(t,s2)\ \ \ \ t \in \{"int", "bool", "string"\}}     
  {\\env \vartriangleright{} \bold{Difference(e_1,e_2)} \implies Set(t,\{s_1 \setminus s_2\}) }  
\subsubsection{Operazioni aggiuntive}
\infr
  {\\env \vartriangleright{} e_1 \implies Set(t,s), env \vartriangleright{} e_2 \implies v:t\ \ \ \ t \in \{"int", "bool", "string"\}}     
  {\\env \vartriangleright{} \bold{Insert(e_1,e_2)} \implies Set(t, s \cup \{v\}) }  
~\\~\\
\infr
  {\\env \vartriangleright{} e_1 \implies Set(t,s), env \vartriangleright{} e_2 \implies v:t\ \ \ \ t \in \{"int", "bool", "string"\}}     
  {\\env \vartriangleright{} \bold{Remove(e_1,e_2)} \implies Set(t, s \setminus \{v\}) }  
~\\~\\~\\~\\
\infr
  {\\env \vartriangleright{} e_1 \implies Set(t,s), env \vartriangleright{} e_2 \implies v:t\ \ \ \ t \in \{"int", "bool", "string"\}}     
  {\\env \vartriangleright{} \bold{Contains(e_1,e_2)} \implies Bool(v \in s) }  
~\\~\\~\\~\\
\infr
  {\\env \vartriangleright{} e_1 \implies Set(t,s), env \vartriangleright{} e_2 \implies v:t\ \ \ \ t \in \{"int", "bool", "string"\}}     
  {\\env \vartriangleright{} \bold{Subset(e_1,e_2)} \implies Bool(\forall{x} \in s) }  
~\\~\\~\\~\\
\infr
  {\\env \vartriangleright{} e \implies Set(t,s)\ \ \ \ t \in \{"int", "bool", "string"\}}     
  {\\env \vartriangleright{} \bold{IsEmpty(e)} \implies Bool(s = \emptyset) }  
~\\~\\~\\~\\
\infr
  {\\env \vartriangleright{} e \implies Set(t,s)\ \ \ \ t \in \{"int", "bool", "string"\}}     
  {\\env \vartriangleright{} \bold{MinOf(e)} \implies min(s):t  }
~\\~\\~\\~\\
\infr
  {\\env \vartriangleright{} e \implies Set(t,s)\ \ \ \ t \in \{"int", "bool", "string"\}}     
  {\\env \vartriangleright{} \bold{MaxOf(e)} \implies max(s):t }
~\\~\\~\\~\\
\infr
  {\\env \vartriangleright{} e_1 \implies Set(t,s), env \vartriangleright{} e_2 \implies f: t \xrightarrow{} Boolean\ \ \ \ t \in \{"int", "bool", "string"\}}     
  {\\env \vartriangleright{} \bold{ForAll(e_1,e_2)} \implies Bool(\forall{x} \in s, Apply(e_2,x) \implies Bool(true))}  
~\\~\\~\\~\\
\infr
  {\\env \vartriangleright{} e_1 \implies Set(t,s), env \vartriangleright{} e_2 \implies f: t \xrightarrow{} Boolean\ \ \ \ t \in \{"int", "bool", "string"\}}     
  {\\env \vartriangleright{} \bold{Exists(e_1,e_2)} \implies Bool(\exists{x} \in s, Apply(e_2,x) \implies Bool(true))}  
~\\~\\~\\~\\
\infr
  {\\env \vartriangleright{} e_1 \implies Set(t,s), env \vartriangleright{} e_2 \implies f: t \xrightarrow{} Boolean\ \ \ \ t \in \{"int", "bool", "string"\}}     
  {\\env \vartriangleright{} \bold{Filter(e_1,e_2)} \implies Set(t,\{x \in s, Apply(e_2,x) \implies Bool(true)\}}  
~\\~\\~\\~\\
\infr
  {\\env \vartriangleright{} e_1 \implies Set(t_1,s), env \vartriangleright{} e_2 \implies f: t_1 \xrightarrow{} t_2\ \ \ \  t_1,t_2 \in \{"int", "bool", "string"\}}     
  {\\env \vartriangleright{} \bold{Map(e_1,e_2)} \implies Set(t_2,\{Apply(e_2,x \in s)\}}  
~\\~\\~\\~\\


\end{document}
